
\chapter{Modified API}

\newcommand{\apidoc}[6]
{
    \subsubsection{\label{api:#1}#2}
    \texttt{#4}\\
    \vspace{3pt}\\
    \noindent
    #3\\
    \begin{table}[h!]
    \rowcolors{2}{gray!25}{white}
    \begin{tabular}{p{0.2\textwidth} p{0.2\textwidth} p{0.6\textwidth}}\toprule
        \emph{Type} & \emph{Parameter} & \emph{Description} \\\midrule
         #5
        \bottomrule
    \end{tabular}
    \end{table}\\
    \noindent
    \textit{Return value:} #6 
}

\newcommand{\param}[3]
{
\texttt{#1} & \texttt{#2} & #3 \\
}

\section{New Invocations}
\subsection{Scheduling context}
\label{api:sc}
\apidoc
{schedcontext_bind}
{SchedContext - Bind}
{ Bind a scheduling context to a provided TCB or Notification object. None of the objects (SC, TCB
or Notification) can be already bound to other objects. If the TCB is in a runnable state and the
scheduling context has available budget, this operation will place the TCB in the scheduler at the
TCB's priority.}
{seL4\_Error seL4\_SchedContext\_Bind}
{
    \param{seL4\_CPtr}{sc}{Capability to the SC to bind an object to.}
    \param{seL4\_CPtr}{cap}{Capability to a TCB or Notification object to bind to this SC.}
}
{0 on success, seL4\_Error code on error.}

\apidoc
{schedcontext_unbind}
{SchedContext - Unbind}
{ Remove any objects bound to a specific scheduling context.}
{seL4\_Error seL4\_SchedContext\_Unbind}
{
    \param{seL4\_CPtr}{sc}{Capability to the SC to unbind an object to.}
}
{0 on success, seL4\_Error code on error.}

\pagebreak
\apidoc
{schedcontext_unbind_object}
{SchedContext - UnbindObject}
{ Remove a specific object bound to a specific scheduling context.}
{seL4\_Error seL4\_SchedContext\_UnbindObject}
{
    \param{seL4\_CPtr}{sc}{Capability to the SC to unbind an object to.}
    \param{seL4\_CPtr}{cap}{Capability to a TCB or Notification object to unbind from this SC.}
}
{0 on success, seL4\_Error code on error.}

\apidoc
{schedcontext_consumed}
{SchedContext - Consumed}
{ Return the amount of time used by this scheduling context since this function
    was last called or a timeout fault triggered.}
{seL4\_Error seL4\_SchedContext\_Consumed}
{
    \param{seL4\_CPtr}{sc}{Capability to the SC to act on.}
}
{An error code and a \code{uint64\_t} consumed value.}

\apidoc
{schedcontext_yieldto}
{SchedContext - YieldTo}
{ 
   If a thread is currently runnable and running on this scheduling context and the scheduling context has available budget, place it at the head of the scheduling queue.
              If the caller is at an equal priority to the thread this will result in the thread being scheduled.
              If the caller is at a higher priority the thread will not run until the threads priority is the highest priority in the system.
              The caller must have a maximum control priority greater than or equal to the threads priority.
}
{seL4\_Error seL4\_SchedContext\_Consumed}
{
    \param{seL4\_CPtr}{sc}{Capability to the SC to act on.}
}
{An error code and a \code{uint64\_t} consumed value.}



\subsection{Sched\_control}
\apidoc
{schedcontrol_configure}
{SchedControl - Configure}
{ Configure a scheduling context by invoking a \code{sched\_control} capability. The scheduling
context will inherit the affinity of the provided \code{sched\_control}.}
{seL4\_Error seL4\_SchedControl\_Configure}
{
    \param{seL4\_CPtr}{sched\_control}{\code{sched\_control} capability to invoke to configure
    the SC.}
    \param{seL4\_CPtr}{sc}{Capability to the SC to configure.}
    \param{uint64\_t}{budget}{Timeslice in microseconds.}
    \param{uint64\_t}{period}{Period in microseconds, if equal to budget, this thread will be
        treated as a round-robin thread. Otherwise, sporadic servers will be used to assure the
    scheduling context does not exceed the budget over the specified period.}
    \param{seL4\_Word}{extra\_refills}{Number of extra sporadic replenishments this scheduling
    context should use. Ignored for round-robin threads. The maximum value is determined by the size
of the SC that is being configured.}
    \param{seL4\_Word}{badge}{Badge value that is delivered to timeout fault handlers}
}
{0 on success, seL4\_Error code on error.}

