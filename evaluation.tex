\chapter{Evaluation}
\label{chap:evaluation}

Questions the evaluation needs to answer:

\begin{enumerate}
\item Show the implementations do not impose undue performance overheads on standard microkernel
operations.
\item Show that criticality mode switch is bounded by high criticality threads.
\item Show the cost of various timeout exception recovery mechanisms.
\item Show that temporal isolation is achieved.
\item Show that deadlines can be met during and after a mode switch.
\item Show the API is pracitcal and consistent with existing frameworks for critical software.
\end{enumerate}

\section{Hardware}

% describe benchmark setup, details of each hardware platform
In addition to several load generators running Linux on experiments are conducted on:

\begin{description}
    \item[ARM:] quad-core 1\,GHz ARM Cortex~A9 system-on-chip on a Freescale i.MX6 SABRE Lite development board,
    \item[x64:] quad-core 3.1\,GHz 64-bit Haswell i7-4770 machine.
\end{description}



\section{Overheads}

We first present a suite of microbenchmarks to evaluate any performane overheads against baseline seL4.


\begin{table}[t]\centering
\begin{tabular}{|c|l| r@{~}l | r@{~}l |r@{~}r|}\hline
\textbf{Arch}           & \multicolumn{1}{c|}{\textbf{Op.}}
                                & \multicolumn{2}{c|}{\textbf{Base}}
                                & \multicolumn{2}{c|}{\textbf{MCS}}
                                & \multicolumn{2}{c|}{\textbf{O/H}} \\ \hline
\multirow{5}{*}{ARM}
\input{data/generated/sabre-micro.inc}
\hline
\multirow{6}{*}{x64}
\input{data/generated/haswell-micro.inc}
& SMP xput & \multicolumn{2}{l|}{1,454,260} & \multicolumn{2}{l|}{1,389,198} & 65k & 4.6\% \\
\hline
\end{tabular}
\caption{Microbenchmarks (cycles) of seL4 baseline vs. MCS kernels,
  standard deviations are shown in brackets. ``*''~indicates an
  average benchmark of 10,000 runs as single measurements showed very
  high variance due to pipeline effects on our hardware.}
\label{t:micro}
\end{table}
